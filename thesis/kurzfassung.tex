\chapter{Kurzfassung}
Diese Bachelorarbeit besch�ftigt sich mit der Aufgabenstellung anhand eines Bildsequenz Objekte zu erkennen und zu verfolgen. Diese Bildsequenz soll anhand einer IP-Kamera aufgezeichnet werden, damit sie �ber das Internet von jedem Ort der Welt ausgewertet werden kann.
Bei den Objekten, die zu erkennen und verfolgen sind handelt es sich um Schiffe und Boote auf einem Fluss. 

Zus�tzlich soll untersucht werden, ob und wenn ja, inwiefern die erkannten Objekte klassifiziert werden k�nnen, um gegebenenfalls Muster im Schiffsverkehr erkennen zu k�nnen. M�gliche Anhaltspunkte f�r Klassifikatoren k�nnen Eigenschaften der Schiffe, wie Typ, Gr��e, und Geschwindigkeit sein.

Der praktische Teil dieser Arbeit soll aus einem Prototypen bestehen, welche den gefunden L�sungsansatz demonstriert. Dieser Prototyp soll aus einer Hardware- und einer Softwarekomponente bestehen.

Die Erkennung, Verfolgung und Klassifizierung soll mittels bekannter und bereits implementierter Computer Vision Algorithmen geschehen und es soll untersucht werden, inwiefern das System m�glichst robust gegen�ber Wettereinfl�ssen erstellt werden kann. Es soll ein theoretischer Ansatz zur Klassifikation der Schiffe entwickelt werden, welche nicht teil der praktischen Arbeit sein soll.