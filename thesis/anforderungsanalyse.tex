\chapter{Anforderungsanalyse}

\section{Anwendungsgebiet}
Die Arbeit besch�ftigt sich mit der Thematik anhand von Bildern und Bildsequenzen ( Videos ) aussagekr�ftige Informationen zu erhalten. Ein Bild repr�sentiert im Computer speicher ist nichts weiter, als eine
Matrix mit bestimmter H�he und Breite, welche multipliziert die Pixelanzahl des Bildes angeben. An jeder Position dieser Matrix sind die Farbwerte zu dem entsprechendem Pixel gespeichert. Durch die Betrachtung dieser Pixelmatrix kann unser menschliches Gehirn ohne weiteres dem Bild Informationen �ber dessen Inhalt entnehmen, folglich das Bild zu interpretieren. Die Interpretation eines Bildes ist ein komplexes Konstrukt und ist durch unsere Erfahrung als Mensch m�glich. Mehr zu diesem Thema in einem abgesonderten Block.
Ein Computer hat allerdings kein komplexes Gehirn und sieht ein Bild, was f�r einen Menschen z.B. gro�en emotionalen Wert haben kann schlichtweg als einen zweidimensionalen Array, welcher konkrete farbwerte enth�lt.
Anhand dieser repr�sentation des Bildes kann allerdings der Computer instruiert werden bestimmte Muster zu erkennen, oder bestimmte Operationen auf das Bild anzuwenden, wodurch letztendlich dem Bild Informationen 
entnommen werden kann. Diese Verarbeitung wird maschinelles Sehen oder Bildverstehen engl. Computer Vision genannt und besch�ftigt sich damit computergest�tzt Bilder auf eine menschliche Art und Weise zu interpretieren. Das Thema maschinelles Sehen wird im Abschnitt Maschinelles Sehen n�her beschrieben.