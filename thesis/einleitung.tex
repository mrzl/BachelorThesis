\chapter{Einleitung}
In den folgenden Kapiteln wird beschrieben, wiö das Problem dieser Bachelor Arbeit angegangen wurde.

\begin{enumerate}
\item \textbf{Einführung }: Was ist die Problem- oder Aufgabenstellung und warum sollte man sich dafür interessieren?
\item \textbf{Analyse / Präzisierung des Themas}:Zunächst wird die Aufgabe näher analysiert. Hier beschreibt man den aktuellen Stand der Technik oder Wissenschaft ("`State-Of-The-Art"'), zeigt bestehende
							Defizite oder offene Fragen auf und entwickelt daraus die Sto\ss{}richtung der eigenen Arbeit.
\item \textbf{Definition}:Daraus ergibt sich eine Definition des Problemes und der naechsten Handlungsschritte.
\item \textbf{Grundlagen}:Dann werden die Grundlagen der Definierten Ziele beschrieben.
\item \textbf{Entwurf / Eigener Ansatz}:Anschliessend wird ein Entwurf des zu entwickelnden Systemes dargelegt.
\item \textbf{Implementierung / Prototyp}:Woraus eine tatsaechliche Implementierung resuliert.
\item \textbf{Test}:Welche getestet werden muss.
\item \textbf{Fazit / Zusammenfassung}:Und durch ein Fazit abgerundet wird.
\end{enumerate}

\section{Motivation}
Für die Entwicklung solche eines Systemes gibts es für den Autor dieser Arbeit diverse Motivationen. Das wichtigste Motiv ist es dem Autor eine Möglichkeit zu geben sich umfangreich und selbstständig mit dem Thema der Computer Vision zu beschäftigen. Die Thematik Informationen über Objekte und deren Zusammenhang anhand von Bildern und Bildersequenzen zu gelangen faszinierte den Autor bereits seit einiger Zeit. Weiterhin hatte der Autor noch nie die Gelegenheit sich mit der Thematik des Maschinellen Lernens auseinanderzusetzen, wo diese Arbeit das Potenzial bietet gro\ss{}en gebrauch davon zu machen.
Im Zusammenhang damit steht, dass der Autor von den Neuigkeiten und Nachrichten bezüglich der \"Uberwachungssysteme im In- und Ausland verwirrt ist und mit eigenen Fähigkeiten herausfinden möchte, inwiefern diese \"Uberwachungssysteme tatsächlich funktionieren und wozu sie in der Lage sind. In den Nachrichten wurde in den letzten Jahren verhäuft über Städte(London) und Situationen(Terroranschlag Boston) berichtet. Dazu möchte der Autor herausfinden, wozu solche Systeme technisch in der Lage sind, und was für einen potenziellen Gewinn an Sicherheit sie versprechen können.

\"Uber diese persönliche Motivation hinaus bietet die Entwicklung eines solchen Systemes gro\ss{}en \"Okonomischen Wert, da dadurch Statistiken und Informationen über den Schiffsverkehr an unterschiedlichen Stellen des gleichen Flusses verglichen werden können, um daraus Schlüsse über ??? zu ziehen.

\section{Relevanz}
\section{Aufbau der Arbeit}