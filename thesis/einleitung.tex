\chapter{Einf\"uhrung}
\begin{quote}
\emph{"`Of all the mighty world of eye and ear, --both what they half create, and what perceive"'\footnotemark}
\end{quote}
\footnotetext{William Wordsworth- Lines Composed a Few Miley Above Tintern Abbey (1798) z.105}

In diesem Abschnitt soll der/die geneigte Leser/in an die Thematik, sowie den Aufbau dieser Arbeit herangef\"uhrt werden. Anspruch des Autors an diese Arbeit ist, ebenfalls Leser mit einem akademischen Spezialisierung in einem Fach weit dem des Autors entfernt, die M\"oglichkeit des vollen Verst\"andnisses der Thematik zu geben.

\section{Motivation}
Die Thematik des Maschinellen Sehens hat den Autor der vorliegenden Arbeit bereits seit einiger Zeit sehr interessiert. Die Tatsache, dass anhand von Bildern und Bildsequenzen automatisch \emph{errechnet} werden kann, was auf dem Bild im menschlichen Verständnis zu sehen ist begeistert den Autor. Durch diese Thematik und Forschungsrichtung ist eine \emph{natürliche} Mensch-Maschine Interaktion erst möglich geworden, wodurch wiederum die Technik des Computers weniger maschinell erscheinen lässt, und eine Art Intelligenz oder Menschlichkeit suggeriert. 
Diese Arbeit gibt dem Autor die Möglichkeit sich intensiv mit der Thematik des Maschinellen Sehens und damit in Verbindung stehenden Techniken auseinanderzusetzen.
Eine weitere und sehr wichtige Motivation sich mit dieser Thematik auseinanderzusetzen ist von sozialer und politischer Natur. Der Autor ist von den fast wöchentlich neu erscheinenden Schlagzeilen, die über Kameraüberwachungssysteme in aller Welt berichten verwirrt. Er möchte selbst ergründen, welche in den Nachrichten benannten Angstszenarien tatsächlich technisch möglich sind. Die Frage, ob eine totale und individuelle \"Uberwachung von Personen anhand von Kamerasystemen im öffentlichen Raum wirklich realistisch ist, beschäftigt den Autor sehr und er hat den Anspruch sich nicht auf oberflächliche Aussagen aus Zeitungsartikeln zu verlassen. Weiterhin möchte er sich technologisches Wissen aneignen, was ihn dazu befähigt begründete Aussagen darüber zu treffen, ob und inwiefern \"Uberwachungssysteme anhand von Videokameras zur erhöhten Sicherheit an öffentlichen Plätzen beiträgt.

\"Uber diese persönlichen und allgemeinen Motivationen hinaus bietet das System, was für die Problemlösung dieser Arbeit erstellt wird durchaus wirtschaftlichen Wert, da dadurch Statistiken und Informationen über den Schiffsverkehr erschlossen werden können.

\section{Aufbau der Arbeit}
Hier wird über den Aufbau der Arbeit zu sprechen gekommen. Es wird erläutert inwiefern die Arbeit strukturiert ist.

\subsection{Einführung}
Es wird eine allgemeine Einleitung in die Thematik und die Beweggründe des Autors zur Erstellung dieser Arbeit gegeben, sowie die Struktur der Arbeit erläutert.
\subsection{Anforderungsanalyse}
In diesem Abschnitt wird die Aufgabenstellung im Detail analysiert. Es wird untersucht, welche Anforderungen an den Prototypen, sowie an die fachliche Ausarbeitung in dieser Arbeit gestellt werden. Es werden die Abgrenzungskriterien benannt, sowie die methodische Vorgehen des Autors zur Bearbeitung der Aufgabenstellung dargelegt.
\subsection{Grundlagen}
Hier werden sämtliche technische Grundlagen, die zum vollen Verständnis des in den folgenden Kapiteln beschriebenen Lösungsansatzes nötig sind beschrieben. In diesem Abschnitt wurde vorsätzlich auf sämtliche mathematischen Herleitungen verzichtet, da es den Umfang dieser Arbeit sprengen würde.
\subsection{Definition}
Im Abschnitt der Definition soll die Herangehensweise des Autors an die Lösung des Problems geschildert werden. Es werden die nötigen Techniken benannt und deren Arbeitsweise erläutert, sowie das Problem näher definiert, woraus die Handlungsschritte der nächsten Kapitel resultieren.
\subsection{Entwurf}
Die Beschreibung des Entwurfes stellt den Lösungsansatzes des Autors dar. Es wird ein Entwurf des zu entwickelnden Prototyps darstellt und erläutert.
\subsection{Implementierung}
Hier wird auf die tatsächliche praktische Implementierung des Systems eingegangen. Es werden spezifische Aspekte der Umsetzung der theoretischen Ansätze aus den vorangegangenen Kapiteln beschrieben und beurteilt.
\subsection{Test}
Das System muss auf dessen Funktionalität getestet werden. Diese Tests werden in diesem Kapitel näher erläutert.
\subsection{Fazit}
Hier wurde eine nachträgliche Betrachtung und Beurteilung der Bearbeitung des Problems dieser Arbeit erarbeitet, welche sämtliche Aspekte des Arbeit aufarbeitet.
\subsection{Zukunftsaussichten}
In diesem Abschnitt werden eventuelle zukünftige Schritte zu Weiterentwicklung des Systems benannt, beschrieben, erläutert und beurteilt.